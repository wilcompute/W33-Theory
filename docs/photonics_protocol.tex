\documentclass[11pt]{article}
\usepackage[margin=1in]{geometry}
\usepackage{iftex}
\ifPDFTeX
  \usepackage[T1]{fontenc}
  \usepackage[utf8]{inputenc}
  \usepackage{textcomp}
  \usepackage{lmodern}
\else
  \usepackage{fontspec}
  \setmainfont{Latin Modern Roman}
  \setsansfont{Latin Modern Sans}
  \setmonofont{Latin Modern Mono}
\fi
\DeclareUnicodeCharacter{2080}{\textsubscript{0}}
\DeclareUnicodeCharacter{2081}{\textsubscript{1}}
\DeclareUnicodeCharacter{2082}{\textsubscript{2}}
\DeclareUnicodeCharacter{2083}{\textsubscript{3}}
\DeclareUnicodeCharacter{2084}{\textsubscript{4}}
\DeclareUnicodeCharacter{2085}{\textsubscript{5}}
\DeclareUnicodeCharacter{2086}{\textsubscript{6}}
\DeclareUnicodeCharacter{2087}{\textsubscript{7}}
\DeclareUnicodeCharacter{2088}{\textsubscript{8}}
\DeclareUnicodeCharacter{2089}{\textsubscript{9}}
\DeclareUnicodeCharacter{2212}{-}
\DeclareUnicodeCharacter{00D7}{\times}
\DeclareUnicodeCharacter{2070}{\textsuperscript{0}}
\DeclareUnicodeCharacter{2071}{\textsuperscript{i}}
\DeclareUnicodeCharacter{00B9}{\textsuperscript{1}}
\DeclareUnicodeCharacter{00B2}{\textsuperscript{2}}
\DeclareUnicodeCharacter{00B3}{\textsuperscript{3}}
\DeclareUnicodeCharacter{2074}{\textsuperscript{4}}
\DeclareUnicodeCharacter{2075}{\textsuperscript{5}}
\DeclareUnicodeCharacter{2076}{\textsuperscript{6}}
\DeclareUnicodeCharacter{2077}{\textsuperscript{7}}
\DeclareUnicodeCharacter{2078}{\textsuperscript{8}}
\DeclareUnicodeCharacter{2079}{\textsuperscript{9}}
\DeclareUnicodeCharacter{207A}{\textsuperscript{+}}
\DeclareUnicodeCharacter{207B}{\textsuperscript{-}}
\DeclareUnicodeCharacter{2190}{\ensuremath{\leftarrow}}
\DeclareUnicodeCharacter{2192}{\ensuremath{\rightarrow}}
\DeclareUnicodeCharacter{2194}{\ensuremath{\leftrightarrow}}
\DeclareUnicodeCharacter{03B1}{\ensuremath{\alpha}}
\DeclareUnicodeCharacter{03B2}{\ensuremath{\beta}}
\DeclareUnicodeCharacter{03B3}{\ensuremath{\gamma}}
\DeclareUnicodeCharacter{03B4}{\ensuremath{\delta}}
\DeclareUnicodeCharacter{03B5}{\ensuremath{\epsilon}}
\DeclareUnicodeCharacter{03B6}{\ensuremath{\zeta}}
\DeclareUnicodeCharacter{03B7}{\ensuremath{\eta}}
\DeclareUnicodeCharacter{03B8}{\ensuremath{\theta}}
\DeclareUnicodeCharacter{03BB}{\ensuremath{\lambda}}
\DeclareUnicodeCharacter{03BC}{\ensuremath{\mu}}
\DeclareUnicodeCharacter{03BD}{\ensuremath{\nu}}
\DeclareUnicodeCharacter{03BE}{\ensuremath{\xi}}
\DeclareUnicodeCharacter{03C0}{\ensuremath{\pi}}
\DeclareUnicodeCharacter{03C1}{\ensuremath{\rho}}
\DeclareUnicodeCharacter{03C3}{\ensuremath{\sigma}}
\DeclareUnicodeCharacter{03C4}{\ensuremath{\tau}}
\DeclareUnicodeCharacter{03C6}{\ensuremath{\phi}}
\DeclareUnicodeCharacter{03C7}{\ensuremath{\chi}}
\DeclareUnicodeCharacter{03C8}{\ensuremath{\psi}}
\DeclareUnicodeCharacter{03C9}{\ensuremath{\omega}}
\DeclareUnicodeCharacter{0394}{\ensuremath{\Delta}}
\DeclareUnicodeCharacter{0398}{\ensuremath{\Theta}}
\DeclareUnicodeCharacter{039B}{\ensuremath{\Lambda}}
\DeclareUnicodeCharacter{03A0}{\ensuremath{\Pi}}
\DeclareUnicodeCharacter{03A3}{\ensuremath{\Sigma}}
\DeclareUnicodeCharacter{03A6}{\ensuremath{\Phi}}
\DeclareUnicodeCharacter{03A8}{\ensuremath{\Psi}}
\DeclareUnicodeCharacter{03A9}{\ensuremath{\Omega}}
\usepackage{microtype}
\usepackage{graphicx}
\usepackage{array}
\usepackage{calc}
\usepackage{longtable}
\usepackage{booktabs}
\usepackage{amsmath,amssymb}
\usepackage{fancyhdr}
\usepackage{hyperref}
\usepackage{fancyvrb}
\usepackage{framed}
\usepackage{xcolor}
\usepackage{tikz}
\newcounter{none}
\newcommand{\real}[1]{#1}

\definecolor{accent}{HTML}{0B3D91}
\definecolor{accent2}{HTML}{111111}
\definecolor{soft}{HTML}{F5F7FB}
\definecolor{shadecolor}{HTML}{F0F3F8}

\hypersetup{
  colorlinks=true,
  linkcolor=accent,
  urlcolor=accent,
  citecolor=accent,
  pdfauthor={},
  pdftitle={Witting/W33 Photonics Protocol}
}

\providecommand{\tightlist}{%
  \setlength{\itemsep}{0pt}\setlength{\parskip}{0pt}}

\setlength{\parskip}{0.6em}
\setlength{\parindent}{0pt}
\renewcommand{\baselinestretch}{1.05}

\pagestyle{fancy}
\fancyhf{}
\fancyhead[L]{\small\textsc{Witting/W33 Photonics Protocol}}
\fancyhead[R]{\small\textsc{W33 / E8}}
\fancyfoot[C]{\thepage}

\newcommand{\theorembox}[1]{%
  \begin{center}
  \setlength{\fboxsep}{8pt}%
  \fcolorbox{accent}{soft}{\parbox{0.9\linewidth}{#1}}%
  \end{center}
}

\newcommand{\subtitletext}{24‑basis KS + Z₃ Pancharatnam Phase}
\newcommand{\doctitle}{Witting/W33 Photonics Protocol}
\newcommand{\docdate}{January 28, 2026}

\begin{document}
\begin{titlepage}
  \vspace*{1.5cm}
  {\Huge\bfseries\color{accent}\doctitle\par}
  \vspace{0.6cm}
  {\Large\color{accent2}\subtitletext\par}
  \vspace{1.0cm}
  \theorembox{\textbf{Claim:} The W33 generalized quadrangle encodes the Standard Model structure via a finite geometric backbone and an explicit E8 root correspondence.}
  \vfill
  {\large\docdate\par}
  \vspace{0.8cm}
  \begin{flushright}
    \textsc{W33 Theory of Everything}\par
    \textsc{Computed Proof + Artifacts}
  \end{flushright}
\end{titlepage}

\tableofcontents
\newpage

\section{Witting/W33 Photonics
Protocol}\label{wittingw33-photonics-protocol}

\subsection{1. Objective}\label{objective}

This protocol tests two \textbf{falsifiable signatures} of the
Witting/W33 structure:

\begin{enumerate}
\def\labelenumi{\arabic{enumi}.}
\tightlist
\item
  \textbf{State‑independent contextuality} via the 24‑basis KS
  inequality (bound 23 vs quantum 24).
\item
  \textbf{\(Z_{3}\) geometric phase} via Pancharatnam/Berry phase loops
  on explicit Witting‑ray triangles.
\end{enumerate}

\subsection{2. KS Inequality (24‑Basis
Subset)}\label{ks-inequality-24basis-subset}

\begin{itemize}
\tightlist
\item
  \textbf{Noncontextual bound:} 23 / 24
\item
  \textbf{Quantum prediction:} 24 / 24 (state‑independent)
\end{itemize}

Docs: - \texttt{docs/witting\_24basis\_inequality.md} -
\texttt{docs/witting\_24basis\_runsheet.md}

\textbf{Noise threshold (depolarizing):} - Visibility \textbf{v
\ensuremath{\geq} 0.944444} (noise p \ensuremath{\leq} 0.055556) -
\texttt{docs/witting\_24basis\_noise\_threshold.md}

\subsection{3. State Preparation}\label{state-preparation}

Two equivalent paths:

\textbf{(A) Direct unitary preparation} -
\texttt{docs/witting\_24basis\_unitaries.json}

\textbf{(B) Optical decomposition} - MZI schedule:
\texttt{docs/witting\_24basis\_mzi\_schedule.md} - Waveplates (rad):
\texttt{docs/witting\_24basis\_waveplates.md} - Waveplates (deg):
\texttt{docs/witting\_24basis\_waveplates\_deg.md}

\subsection{4. KS Measurement Run‑Sheet}\label{ks-measurement-runsheet}

Use the basis order and ray definitions in: -
\texttt{docs/witting\_24basis\_runsheet.md}

Each basis uses four orthogonal rays. The score S is the number of bases
with exactly one designated outcome.

\subsection{\texorpdfstring{5. \(Z_{3}\) Pancharatnam Phase
Test}{5. Z\_\{3\} Pancharatnam Phase Test}}\label{z_3-pancharatnam-phase-test}

\textbf{Signature:} phases clustered at 0 and
\ensuremath{\pm}2\ensuremath{\pi}/3.

\begin{itemize}
\tightlist
\item
  Example triangles: \texttt{docs/witting\_pancharatnam\_examples.md}
\item
  Full run‑sheet: \texttt{docs/witting\_pancharatnam\_runsheet.md}
\item
  Measurement protocol: \texttt{docs/witting\_pancharatnam\_protocol.md}
\end{itemize}

\subsection{6. Implementation Checklist}\label{implementation-checklist}

\begin{itemize}
\tightlist
\item
  Calibrate phase reference across all interferometric measurements.
\item
  Verify orthonormality of each basis (unitary columns).
\item
  Collect counts for all 24 bases \ensuremath{\rightarrow} compute KS
  score.
\item
  Measure triangle phases for \(Z_{3}\) signature.
\end{itemize}

\subsection{7. Summary of Expected
Outcomes}\label{summary-of-expected-outcomes}

\begin{itemize}
\tightlist
\item
  KS violation: \textbf{S = 24}, bound \textbf{S \ensuremath{\leq} 23}.
\item
  \(Z_{3}\) phase quantization: \textbf{Φ \ensuremath{\in} \{0,
  \ensuremath{\pm}2\ensuremath{\pi}/3\}}.
\end{itemize}

If either fails, the Witting/W33 photonic realization is falsified.

\IfFileExists{latex/external_sources.tex}{
  \clearpage
  \input{latex/external_sources.tex}
}{}

\end{document}
